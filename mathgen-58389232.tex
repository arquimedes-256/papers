

\documentclass[11pt]{amsart}
\usepackage{amsfonts}
\usepackage{amsmath}
\usepackage{amsthm}
\usepackage{amssymb}
\usepackage{mathrsfs}
\usepackage[numbers]{natbib}
\usepackage[fit]{truncate}
\usepackage{fullpage}

\newcommand{\truncateit}[1]{\truncate{0.8\textwidth}{#1}}
\newcommand{\scititle}[1]{\title[\truncateit{#1}]{#1}}

\theoremstyle{plain}
\newtheorem{theorem}{Theorem}[section]
\newtheorem{corollary}[theorem]{Corollary}
\newtheorem{lemma}[theorem]{Lemma}
\newtheorem{claim}[theorem]{Claim}
\newtheorem{proposition}[theorem]{Proposition}
\newtheorem{question}{Question}
\newtheorem{conjecture}[theorem]{Conjecture}
\theoremstyle{definition}
\newtheorem{definition}[theorem]{Definition}
\newtheorem{example}[theorem]{Example}
\newtheorem{notation}[theorem]{Notation}
\newtheorem{exercise}[theorem]{Exercise}
\newtheorem{step}[theorem]{Step}
\begin{document}


\begin{abstract}
I present a unconditional probabilistic operator that determines abelian terms from single input element using abstract structure as search context.
\end{abstract}


\scititle{Term-search operator for abelian group}
\author{Lucas Oliveira}
\date{}
\maketitle











\section{Introduction}


...


\section{Principle}

\iffalse
Introduzir permutação de operador
utilizar algebra de lie
\fi

\begin{definition}
 Let $\varphi : \langle X, G \rangle \mapsto \mathcal L^*$ where $\mathcal L^* \subset \mathcal L = \{\ell_1,\cdots \ell_{|\mathcal L|}\}$
\end{definition}

\begin{definition}
	Let $G\langle \mathcal L , \circ,e \rangle $ a abelian group and 
	$X \in 
		\{
		\mathcal L
		\cup
		\sup \mathcal L 
		\cup  
		\inf \mathcal L
		\}
		$ 
		such that $X = \ell_* \circ \ell_{*'} \circ \cdots \circ \ell_{*''}$ where each $\ell_* $ is unknow apriori and lives in image of $\varphi \mapsto \mathcal L^*$

\end{definition}

\begin{definition}
	Let $
	\mathcal M:\mathcal L_k 
		\mapsto 
		\{\mathcal L\} $
	 a one-to-many map, mapping each element result of a comutation $k$ living or not in $\mathcal L$ to a family of possible(s) comutation(s) term(s) set.
	 A thing like:
	 	$$
	 	\mathcal M:
	 	(\ell_k \circ \ell_{k'} \circ \cdots \circ \ell_{*''}) \mapsto 
	 	\Big\{
	 	\{\ell_k ,\ell_{k'} , \cdots , \ell_{k''}\},\cdots
	 	\Big\}
	 	$$
	 The structure of $\mathcal M$ can be imaginated as bi-dimensional matrix, but you cannot define a priori your size because it is $k$-iteration-dependent.
\end{definition}




\begin{step}
Determine $\mathcal M(\ell_i) \leftarrow \ell_i$ for $i=1..|\mathcal L|$
\end{step}

\begin{step}
	Let $k$ a iteration block ; Pick $\ell_\alpha$ , $\ell_\beta$ sample pair elements from $\mathcal L$ 
	
	$$
	\mathcal M(\ell_\alpha \circ \ell_\beta) 
		\leftarrow 
			\{ \ell_\alpha , \ell_\beta \} 
		\iff
		(k=0) 
		\lor
			(|\mathcal M| = |\mathcal L| )
	$$

	
\end{step}
\begin{step}
		A trivial step for each iteration $k$ before sample the pair $\{ \ell_\alpha , \ell_\beta \}$ is compute: 
		$$
		\texttt{if}
		\Big(
			(\ell_\alpha \circ \ell_\beta)^{-1} 
			\circ X = e 
		\Big) 
		\texttt{then return }
			X=  \{ \ell_\alpha , \ell_\beta \}	
		$$
\end{step}

\begin{step}
	Iff evaluation is diferent than $e$ :
	 
	A second step is check if the inverse exists in map $\mathcal M$ , computing: $$\mathcal L_\mathcal M \leftarrow 
	\mathcal M
	\Big(
		(\ell_\alpha \circ \ell_\beta)^{-1} 
		\circ X
	\Big)$$
	Before check
	$$
	\texttt{if}
	\Big((\ell_\alpha \circ \ell_\beta\circ\mathcal L_\mathcal M)^{-1} \circ X = e 
	\Big) \texttt{then return }X=  \{ \ell_\alpha , \ell_\beta \}	\cup \mathcal L_\mathcal M
	$$	
\end{step}

\begin{step}
	In this step $X_k^{-1}$ denote $(\ell_\alpha \circ \ell_\beta\circ\mathcal L_\mathcal M)^{-1}$
	then
		$$
		\texttt{if}
		\Big(
			X_k^{-1} \circ X \neq e 
			\land
			\exists \,
\mathcal M (\mathcal L_\mathcal M)			
		\Big) 
		\texttt{then push }
		\{ \ell_\alpha , \ell_\beta \}
		\cup
		\mathcal L_\mathcal M 
		\rightarrow
		\mathcal M
		\Big(
			( \ell_\alpha \circ \ell_\beta )
		\circ 
			\mathcal L_\mathcal M 
		\Big)
		$$
\end{step}
\begin{step}
	Compute next $k$
\end{step}
\section{Conclusion}

\cite{cite:30}


\begin{footnotesize}
\bibliography{scigenbibfile}
\bibliographystyle{plainnat}
\end{footnotesize}

\end{document}
